\chapter{Blatt 07}
\section{Aufgabe 1}
Sei also $(x_n)_{n\in \mathbb{N}}$ eine Cauchyfolge. Dann
\begin{align}
  \forall\epsilon > 0\quad :\quad \exists N \in \mathbb{N}\quad :\quad |x_m - x_n|<\epsilon \quad \forall m,n \geq N
\end{align}
Dies gilt natürlich für $n=N$, also
\begin{align}
  |X_m - X_n|<\epsilon \quad \forall m \geq N
\end{align}
Somit
\begin{align}
  ||X_m| - |X_n||<\epsilon
\end{align}
\textbf{i)} Ist $|x_n| - |x_N|\geq 0$
\section{Aufgabe 2}
\section{Aufgabe 3}
\section{Aufgabe 4}
