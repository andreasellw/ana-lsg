\chapter{Blatt 04}
\section{Aufgabe 1}
\textit{Bemerkung:} Archimedisches Axiom, Bernoulli Ungleichung\\
\begin{align}
  0<q<1
\end{align}
zu Zeigen:
\begin{align}
  \forall \varepsilon >0\quad :\quad\exists N\in\mathbb{N}\quad :\quad q^n<\varepsilon \quad\forall n>N
\end{align}
Archimedisches Axiom:
\begin{align}
  \forall \varepsilon >0\quad :\quad\exists N\in\mathbb{N}\quad :\quad0<\frac{1}{\varepsilon}<1+nx,\quad x > 0
\end{align}
Bernoulli Ungleichung:
\begin{align}
  (1+x)^n\geq 1+nx \qquad\forall x\geq-1,\quad\forall n\geq0
\end{align}
\begin{align}
  \frac{1}{\varepsilon}\overset{Ar.}{<}1+nx\overset{Be.}{\leq}(1+x)^n
\end{align}
Damit
\begin{align}
  \frac{1}{\varepsilon}<(1+x)^n \Leftrightarrow \frac{1}{(1+x)^n}<\varepsilon
\end{align}
Setze $x:= \frac{1}{q}-1$, dann folgt
\begin{align}
  \frac{1}{(\frac{1}{q})^n}<\varepsilon \Leftrightarrow q^n<\varepsilon\qed
\end{align}
\section{Aufgabe 2}
\textit{Bemerkung:} Cantor Diagonalargument\\
zu a)\\
Sei $M_j$ $\forall j \in \mathbb{N}$ eine abzählbare Menge. Dann ist zu zeigen: $\bigcup\limits_{j \in \mathbb{N}}M_j$ ist abzählbar. Da $M_j$ abzählbar, existiert für jedes $j\in\mathbb{N}$ bijektive Abbildung $\rho_j:\mathbb{N}\rightarrow M_j$.\\
Nummeriere Elemente von $M_j$ wie folgt:
\begin{align}
  m_{1}^{j}\quad : \quad \rho_{j}(1) ... m_{n}^{j}:= \rho_{j}(n)
\end{align}
\section{Aufgabe 3}
\section{Aufgabe 4}
