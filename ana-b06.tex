\chapter{Blatt 06}
\section{Aufgabe 1}
\section{Aufgabe 2}
\begin{flalign}
  a_n=\frac{2n+1}{n+2},\quad b_n=1+\frac{(-1)^n}{n}+\frac{1}{n^2}\qquad\forall n \in \mathbb{N} &&&
\end{flalign}
\begin{enumerate}[label={\alph*)}]
  \item
  \begin{flalign}
    &a_1=1, \quad a_2=\frac{5}{4}, \quad a_3=\frac{7}{5}, \quad a_4=\frac{9}{6}, \quad a_5=\frac{11}{7}&& \nonumber \\
    &b_1=1, \quad b_2=\frac{7}{4}, \quad b_3=\frac{7}{9}, \quad b_4=\frac{21}{16}, \quad b_5=\frac{21}{25}&& \nonumber
  \end{flalign}
  \item
  Da $b_1<b_2$ aber $b_2>b_3$ kann $b_n$ nicht monoton sein.\\
  $a_n$ monoton steigend?
  \begin{flalign}
    a_n<a_{n+1}\qquad \forall n &&&
  \end{flalign}
  Es gilt:
  \begin{flalign}
    &a_n=\frac{2n+1}{n+2}=\frac{2(n+1)-3}{n+2}=2- \frac{3}{n+2} \Longrightarrow a_{n+1}=2- \frac{3}{n+3} && \\
    &n+2<n+3 \Longrightarrow \frac{1}{n+2}>\frac{1}{n+3} \Longrightarrow -\frac{1}{n+2}<-\frac{1}{n+3} \Longrightarrow 2-\frac{3}{n+2}<2-\frac{3}{n+3}\qquad \forall n &&
  \end{flalign}
  \item
  Gibt es $c_a, C_a, c_b, C_b \in \mathbb{R} : \quad c_a\leq a_n \leq C_a$\\
  $a_n \geq 0 \quad \forall n \in \mathbb{N}$, also $c_a=0$.\\
  Außerdem $a_n=2-\frac{3}{n+2}$ auch $a_n\leq 2\quad \forall n$\\ $\Longrightarrow C_a=2$\\
  Es gilt:
  \begin{flalign}
    b_n &=1+ \frac{(-1)^n}{n}+\frac{1}{n^2}=\frac{n^2+(-1)^n\cdot n+1}{n^2} && \nonumber \\
    &\geq\frac{n^2-n+1}{n^2}\geq \frac{n^2-2n+1}{n^2}= \frac{(n-1)^2}{n^2}\geq 0 \qquad \forall n \in \mathbb{N} &&
  \end{flalign}
  $\Longrightarrow c_b=0$\\
  Außerdem $\frac{1}{n}\leq 1$, $\frac{1}{n^2}\leq 1$.\\
  Also
  \begin{flalign}
    b_n &=1+ \frac{(-1)^n}{n}+\frac{1}{n^2}\leq 1+ \frac{1}{n}+\frac{1}{n^2}\leq 3 &&& \nonumber
  \end{flalign}
  $\Longrightarrow C_b=3$\\
  \item
  Konvergenz?\\
  $a_n$: monotone Folgen und beschränkt $\overset{VL}{\Longrightarrow}$ konvergent.\\
  $b_n$: $\frac{1}{n}$, $\frac{1}{n^2}$ sind Nullfolgen und wegen ...
\end{enumerate}
\section{Aufgabe 3}
\section{Aufgabe 4}
